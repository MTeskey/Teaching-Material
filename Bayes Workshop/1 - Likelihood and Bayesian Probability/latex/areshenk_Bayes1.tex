\documentclass{beamer}
\usetheme{simple}

\usepackage{amsmath}
\usepackage{pgfplots}
\usepackage[export]{adjustbox}
\usepackage{lmodern}
\usepackage{transparent}
\usepackage[scale=2]{ccicons}

\setbeamercovered{invisible}

\title{Bayes Workshop 1}
\subtitle{Bayesian Probability and the Likelihood Function}
\date{\today}
\author{Corson N. Areshenkoff}
\institute{University of Victoria}

\begin{document}

\maketitle

\begin{frame}{The Problem}
Find a coin on the sidewalk. How do we know if the coin is fair? \par
In other words, we want to know the probability $p$ that a coin returns heads.
\end{frame}

\begin{frame}{The Problem}
How do we get $p$?
        \begin{itemize}
                \item First, specify a probability model. 
                \begin{itemize}
                        \item If a coin flip is random, what is its distribution?
                        \item If we flip the coin repeatedly, how many heads should we see?
                \end{itemize}     
                \item Then, collect data
                \begin{itemize}
                        \item Flip the coin many times and observe the proportion of heads.
                        \item The number of heads obtained in $N$ independent flips of a coin is a \emph{binomial} random variable.
                \end{itemize}                                           
        \end{itemize}
\end{frame}

\begin{frame}{Probability Mass Function}
\begin{itemize}
        \item The \emph{probability mass function} (pmf) of a (discrete) random variable gives the probability of observing an outcome $x$.
        \item For $N$ coin flips (with probability $p$ of heads), the pmf gives the probability of observing $k$ heads.
\end{itemize}        
\end{frame}

\begin{frame}{Probability Mass Function}
How do we get the pmf? \par Suppose we flip a fair coin 4 times ($N=4, p = 0.5$). What is the probability that we get ONE head?
        \begin{itemize}
                \item The head happens with probability $\frac{1}{2}$
                \item Each tails happens with probability $\frac{1}{2}$, giving
                         $\left ( \frac{1}{2} \right )^3$
                \item \dots but there are $4$ places the head can appear.
        \end{itemize}
So the probability is 
        \[
                P(1;N=4,p=0.5) = 
                        4 \left ( \frac{1}{2} \right ) \left ( \frac{1}{2} \right )^3
        \]
\end{frame}

\begin{frame}{Probability Mass Function}
In general, the binomial pmf is given by
                \[
                P(k;N,p) = \underbrace{{n \choose k}}_{\text{\# of ways}}
                                \overbrace{p^k}^{\text{k heads}}
                                \underbrace{(1-p)^{n-k}}_{\text{n-k tails}}
                \]
\end{frame}

% Likelihood
\begin{frame}{Likelihood}
        \begin{itemize}
                \item When we flip a fair coin $N$ times, we talk about the \emph{probability} of observing $k$ heads ($N$ and $p$ are fixed parameters; the \emph{outcome} is unknown).
                        \begin{itemize}
                                \item Probability mass function $P(k;N,p)$
                        \end{itemize}
                \item After performing the experiment and \emph{observing} $k$ heads, we can treat $p$ as unknown and ask: what is the probability that a specific value of $p$ would give us $k$ heads? (what is the \emph{likelihood} that $p$ takes a specific value?)
                        \begin{itemize}
                                \item Likelihood function $\mathcal{L}(p;N,k)$
                        \end{itemize}
        \end{itemize}
\end{frame}

\begin{frame}{Likelihood}
A practical example:
        \begin{itemize}
                \item Flip a coin $4$ times and observe $3$ heads. Which value of $p$ is
                         most likely?
                \item For $p = 0$, the probability of observing $3$ heads is zero, so
                          so the likelihood that $p=0$ is zero.
                \item For $p = 1/2$, the probability of observing $3$ heads is $0.25$, so
                          so the likelihood that $p=1/2$ is $1/4$.
                \item For $p = 3/4$, the probability of observing $3$ heads is $0.43$, so
                          so the likelihood that $p=3/4$ is $0.43$.
                          \begin{itemize}
                                  \item This is the value of $p$ with the \emph{highest} 
                                            likelihood. In other words, $p = 3/4$ maximizes the 
                                            probability of obtaining our data.
                          \end{itemize}
        \end{itemize}
\end{frame}

\begin{frame}{The Likelihood Function}
        \begin{itemize}
                \item Every statistical model (regression, distribution fitting, etc) produces
                          a \emph{likelihood function}, which gives the probability of 
                          obtaining a set of data for different values of the model parameters.
                \item If $D$ are the data, and $\Theta$ are the parameters of the model 
                         (say, regression coefficients), then
                                 \[ \mathcal{L}(\Theta|D) = P(D|\Theta) \]
                \item Common practice is to choose the parameters values which 
                          maximize the likelihood function (i.e. which maximize the 
                          probability of obtaining the observed data).
                          \begin{itemize}
                                  \item This is \emph{maximum likelihood estimation}, and the
                                            resulting parameter estimates are called the 
                                            \emph{maximum likelihood estimates} (MLE's)
                          \end{itemize}
        \end{itemize}
\end{frame}

\begin{frame}{The Likelihood Function}
Some examples:
        \begin{itemize}
                \item The MLE for $p$ in our coin flip example
                          is the observed proportion of heads $\hat{p} = k/N$.
                \item The MLE's for the mean and variance of a normal distribution are
                        \begin{align*}
                                \hat{\mu} &= \bar{X} = \frac{\sum_{i=1}^N x_i}{N} \\
                                \hat{\sigma}^2 &= \frac{\sum_{i=1}^N (\bar{X} - x_i)^2}{N}
                        \end{align*}
                \item The MLE's for a linear regression model are the usual least squares
                          estimates.
        \end{itemize}
\end{frame}

\begin{frame}{The Likelihood Function}
        \begin{itemize}
                \item The likelihood function is, arguably, the most important concept in
                          statistical inference.
                \item \textbf{All of the information contained in the data about a model is 
                               contained in the likelihood function}.
                \item In practice, the natural logarithm of the likelihood function
                         (the log-likelihood) is usually easier to work with.
        \end{itemize}
\end{frame}


% Bayes
\begin{frame}{Bayesian Estimation}
Suppose that we pick up a newly minted coin and flip it $N=4$ times, obtaining $k=3$ heads.
The MLE for the probability $p$ is $k/N = 0.75$.\par
Which is more plausible?
    \begin{itemize}
        \item The finely tuned machines used to mint the coin made a mistake, and somehow minted a  highly unbalanced coin.
        \item The coin is fair, but just happened to return 3 heads.
    \end{itemize}
\end{frame}

% Fitting models in STAN
\begin{frame}[fragile]
The \textbf{data} block specifies the data that will be passed to Stan:
\begin{verbatim}
    data {
        int<lower=0> N; // Number of coin flips
        int k;          // Number of heads
    }
\end{verbatim}
\end{frame}

\begin{frame}[fragile]
The \textbf{parameters} block specifies the parameters of our model:
\begin{verbatim}
    parameters {
        real<lower=0, upper = 1> p;
    }
\end{verbatim}
\end{frame}

\begin{frame}[fragile]
The \textbf{model} block specifies the likelihood/priors:
\begin{verbatim}
    model {
        h ~ binomial(N, p);  // Likelihood
        p ~ beta(1,1);       // Prior
    }
\end{verbatim}
\end{frame}

\begin{frame}[fragile]
The complete .stan file is then:
\begin{verbatim}
    data {
        int<lower=0> N; // Number of coin flips
        int k;          // Number of heads
    }
    
    parameters {
        real<lower=0, upper = 1> p;
    }
    
    model {
        h ~ binomial(N, p);  // Likelihood
        p ~ beta(1,1);       // Prior
    }
\end{verbatim}
\end{frame}

\end{document}